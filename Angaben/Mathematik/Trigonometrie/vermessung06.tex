\exkw{Winkelfunktionen}
\exkw{Trigonometrie}
\exkw{sin}
\exkw{cos}
\exkw{tan}
\excomp{AG}
\excomp{AG4}
\exquestion{Vermessungsaufgabe}

Ein Mast steht auf einer waagrechten Ebene. In der Ebene wird eine Standlinie AB abgesteckt,
die mit dem Fußpunkt F des Mastes ein Dreieck bildet. Man misst die
Horizontalwinkel $BAF = \psi$, $ABF = \phi$ und den Höhenwinkel $\alpha$ von A zur Mastspitze.
Wie hoch ist der Mast, wenn $\bar{AB} = 80m$, $\psi = 72.8$, $\phi = 38.4$ und $\alpha = 28.3$
beträgt?
