\exkw{Winkelfunktionen}
\exkw{Trigonometrie}
\exkw{sin}
\exkw{cos}
\exkw{tan}
\excomp{AG}
\excomp{AG4}
\exquestion{Vermessungsaufgabe}

Ein gerader Weg steigt von einem Ort A aus gleichmäßig unter dem Höhenwinkel
$\epsilon = 9.8$ bis zum Fußpunkt C einer Felswand an. Die Felswand ist unter
einem Winkel $\gamma = 72.5$ gegen die Horizontale geneigt. Von A aus sieht
man die Spitze S der Wand unter dem Höhenwinkel $\alpha = 33.6$ geht man
$650 m$ näher zur Wand, so sieht man von diesem neuen Punkt B die Spitze
der Wand unter dem Höhenwinkel $\beta = 50.4$.  Berechne die Entfernung des
Punktes B vom Fußpunkt C der Wand, die Länge CS der Wand, die relative
Höhe der Spitze S über C, S über A und C über A.