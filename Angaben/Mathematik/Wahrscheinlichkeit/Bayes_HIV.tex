\exkw{Wahrscheinlichkeit}
\exkw{bedingte Wahrscheinlichkeit}
\exkw{Bayes}
\excomp{WS}
\excomp{WS2}
\exquestion{HIV Test}

Unter der Gesamten Bevölkerung liegt die Infektionsrate mit HIV in der Altersgru
ppe 15-49 bei ca. $0.1\%$. Die Sensititivtät (d.h. die Wahrscheinlichkeit, dass der
Test ein positives Ergebnis liefert wenn eine Person infiziert ist) liegt bei
$99.9\%$. Die Spezifität des ELISA-Tests (d.h. die Wahrscheinlichkeit, dass eine
nicht infizierte Person auch als negativ erkannt wird) liegt bei $99.8\%$.

\begin{itemize}
	\item Wie wahrscheinlich ist es, dass eine Person die ein positives Testergebnis
		erhält tatsächlich infiziert ist?
	\item Wenn der selbe Test ein zweites Mal ausgeführt wird und davon ausgegangen
		werden kann, dass beide Testdurchgänge unabhängig sind (bzw.
		das keine systematischen Fehler vorliegen), wie hoch ist die Wahrscheinlichkeit
		bei zwei positiven Ergebnissen tatsächlich infiziert zu sein?
	\item Wie hoch ist die Wahrscheinlichkeit bei einem negativen Ergebnis bei nur
		einem durchgeführten Test dennoch infiziert zu sein?
\end{itemize}