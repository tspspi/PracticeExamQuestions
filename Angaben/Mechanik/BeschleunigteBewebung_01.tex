\exkw{Bewegung}
\exkw{Beschleunigt}
\exkw{Gravitation}
\exquestion{Beschleunigte Bewegung / Wurf}

Ein Ball mit einer Masse von $m=\fra{1}{2}\text{kg}$ wird in einem Winkel von
$\alpha=35^\circ$ schräg nach oben geworfen. Auf ihn wirkt die Erdbeschleunigung
mit $g\approx10\frac{\text{m}}{\text{s}^2}$.

\begin{itemize}
	\item Stelle die zugehörige Bewegungsgleichung auf
	\item Bestimme nach welcher Zeit der Ball wieder auf den Boden aufschlägt
	\item Bestimme nach welcher Zeit der Ball die maximale Höhe erreicht
	\item Bei welchem Winkel wird die maximale Wurfweite erreicht?
	\item (Optional): Wir würde sich die Bewegungsgleichung ändern, wenn die Reibung
		mit der Atomosphäre nicht vernachlässigt werden würde und diese
		durch Kraft von $F_w \propto c_w * A * \frac{1}{2} * \rho * v^2(t)$ beschrieben
		werden kann, wobei $\rho$ die Dichte des mediums, $A$ die Bezugsfläche des Körpers
		und $c_w$ der Strömungswiederstandskoeffizient ist.
\end{itemize}