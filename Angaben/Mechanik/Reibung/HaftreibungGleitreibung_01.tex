\exkw{Mechanik}
\exkw{Haftreibung}
\exkw{Gleitreibung}
\exquestion{Reibungskoeffizienten (Haft- und Gleitreibung)}

Um den Haftreibung- und Gleitreibungskoeffizienten zwischen zwei Materialien
bestimmen zu können, kann im einfachsten Fall eine Schiefe ebene verwendet
werden. Hierzu wird ein Objekt auf eine Ebene mit variabler Neigung $\alpha$
gegen die Horizontale platziert. Danach wird die Neigung solange erhöht, bis
das Objekt anfängt zu rutschen - nun wird der Neigungswinkel so lange reduziert,
bis das Objekt wieder stillsteht.

\begin{itemize}
	\item Zeige, wie aus diesen Informationen der Haftreibungskoeffizient $\mu_h$
		und der Gleitreibungskoeffizient $\mu_r$ bestimmt werden kann.
	\item Zeige desweiteren, dass die Methode der Bestimmung der Koeffizienten vom
		Betrag der Schwerebeschleunigung (Gravitation) unabhängig ist.
	\item Welchen weiteren Reibungskoeffizienten kennst du?
\end{itemize}