\exkw{CS}
\exkw{Strings}
\exkw{Substring}
\exkw{Console}
\exkw{Dictionary}
\exkw{while}
\exquestion{Worte zählen}

Schreibe ein Programm, dass über die Konsole eine beliebig lange Eingabe akzeptiert. Solange
sie keine zwei aufeinanderfolgenden Leerzeilen einliest oder die Eingabe auf andere Art und Weise
beendet wird, soll das Programm alle auftretenden Worte (jeweils durch beliebige Anzahl an
Whitespace wie Leerzeichen oder Zeilenumbrüche getrennt) zählen.

Um die Worte und Zähler zu speichern kann ein Dictionary eingesetzt werden. Das Dictionary wird mit

\begin{lstlisting}
   Dictionary<string, int> dict = new Dictionary<string, int>();
\end{lstlisting}

erstellt. Danach können Werte wie zu einem Array hinzugefügt oder geändert werden, es kann aber auch
die Add Funktion genutzt werden:

Hinzufügen eines noch nicht existierenden Eintrags:
\begin{lstlisting}
   dict.Add("string", 1);
   // oder
   dict["string"] = 1;
\end{lstlisting}

Ändern eines existierenden Eintrags:
\begin{lstlisting}
   dict["string"]++;
\end{lstlisting}

Um zu prüfen ob ein Eintrag existiert kann ContainsKey verwendet werden:

\begin{lstlisting}
   if(dict.ContainsKey("string")) {
      // Key existiert
   }
\end{lstlisting}

Die Werte können ebenfalls über den selben Weg wie ein Arrayzugriff ausgelesen werden:

\begin{lstlisting}
   int a = dict["string"];
\end{lstlisting}

Um alle Einträge eines Dictionaries anzuzeigen kann ein foreach genutzt werden:

\begin{lstlisting}
   foreach(KeyValuePair<string, int> kvp in dict) {
      string key = kvp.Key;
      int value = kvp.Value;
   }
\end{lstlisting}

Erweitere das Programm nun so, dass die Worthäufigkeit in Prozent dargestellt wird.