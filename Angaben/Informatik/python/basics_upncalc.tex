\exkw{Python}
\exkw{while}
\exkw{if}
\exquestion{UPN Rechner}

In der folgenden Übungsaufgabe soll ein UPN Rechner implementiert werden.
Hierbei kann ein Nutzer Zahlen oder Rechenoperationen eingeben. Wird eine
Zahl eingegeben, wird sie auf einen Stapel, der mit Hilfe einer Liste
implementiert wird gelegt (\texttt{liste.add(...)}). Wird eine Rechenoperation eingegeben,
werden die letzten 2 Einträge von der Liste geholt (das kann mit zwei
aufeinanderfolgenden \texttt{n = liste.pop();} Operationen erfolgen), die entsprechende
Rechenoperation ausgeführt und ihr Ergebnis wiederum mit \texttt{liste.add(...)} an das
Ende der Liste angefügt sowie auf der Standardausgabe angezeigt. Hierbei sollen
prinzipiell die bekannten Operationen ($+$,$-$,$/$,$*$,etc.) implementiert werden.

Sofern möglich soll eine Überprüfung der Listenlänge realisiert werden, um im
Fall einer Rechenoperation die z.B. zwei Argumente voraussetzt zu prüfen, ob
wirklich mindestens zwei Zahlen in der Liste enthalten sind und sonst
eine entsprechende Fehlermeldung angezeigt werden.
